 %------------------------------------------------------------------------------
% Template file for the submission of papers to IUCr journals in LaTeX2e
% using the iucr document class
% Copyright 1999-2013 International Union of Crystallography
% Version 1.6 (28 March 2013)
%------------------------------------------------------------------------------

\documentclass{iucr}              % DO NOT DELETE % 
\usepackage{bm}
% \usepackage{graphicx}
% \usepackage{tabularx}
% \usepackage{subfigure}
% \usepackage{afterpage}
% \usepackage{sansmath}
\usepackage{mathtools}
% \usepackage{parskip}
% \usepackage{tikz}
% \usepackage{tikzorbital}
% \usepackage{setspace}
% \usepackage{xcolor}
% \usepackage{amssymb}
% \usepackage{bm}
\usepackage{amsmath}
% \usepackage{fancyhdr}
% \usepackage{rotating}
% \usepackage{siunitx}
\usepackage[hyphens,spaces,obeyspaces]{url}
\usepackage{color}

\newcommand{\todo}[1]{{\color{red}[TODO: "#1'']}}
\newcommand{\manuel}[1]{{\color{red}[Manuel: #1]}}
\newcommand{\inblue}[1]{{\color{blue}#1}}
\newcommand{\inred}[1]{{\color{red}#1}}
\newcommand{\ingreen}[1]{{\color{green}#1}}



     %-------------------------------------------------------------------------
     % Infobrmation about journal to which submitted
     %-------------------------------------------------------------------------
     \journalcode{S}              % Indicate the journal to which submitted
                                  %   A - Acta Crystallographica Section A
                                  %   B - Acta Crystallographica Section B
                                  %   C - Acta Crystallographica Section C
                                  %   D - Acta Crystallographica Section D
                                  %   E - Acta Crystallographica Section E
                                  %   F - Acta Crystallographica Section F
                                  %   J - Journal of Applied Crystallography
                                  %   M - IUCrJ
                                  %   S - Journal of Synchrotron Radiation

\begin{document} % DO NOT DELETE THIS LINE

     %-------------------------------------------------------------------------
     % The introductory (header) part of the paper
     %-------------------------------------------------------------------------

     % The title of the paper. Use \shorttitle to indicate an abbreviated title
     % for use in running heads (you will need to uncomment it).


\title{Diaboloid mirrors III: simulations and applications}

\cauthor[a,b]{Manuel}{Sanchez del Rio}{srio@esrf.eu}{address if different from \aff}
\author[a]{Valeriy V.}{Yashchuk}
\author[a]{Ian}{Lacey}
\author[a]{Kenneth A.}{Goldberg}
\author[a]{Howard A.}{Padmore}

\aff[a]{Advanced Light Source, LBNL, Berkeley CA, USA}
\aff[b]{European Synchrotron Radiation Facility, 71 Avenue des Martyrs F-38000 Grenoble \country{France}}



\begin{synopsis}
xxxxx
\end{synopsis}

\begin{abstract}
The diaboloid is a reflecting surface that convert a spherical wave to a cylindrical wave. This complex surface may find application in new ALS-U bending magnet beamlines or in other  beamlines that now exploit todoidal optics for astigmatic focusing. We describe here the implementation of the diaboloid mirrors in the Oasys environment with some results of  simulations. 
\end{abstract}

\section{Introduction}

In the development of synchrotron radiation sources, brightness has always been a key metric, and as such the emphasis has always been on undulators.  A secondary consideration has been the development of the bending magnets. These white light sources hold key advantages for a range of x-ray experiments, such as Laue diffraction, dispersive EXAFS and other experiments which take advantage of a wide bandwidth.  In other areas, the agility in photon energy offered by a monochromatized bending magnet radiation is a key advantage.  Now with the advent of 4th generation MBA synchrotron sources, bending magnet sources are becoming even more attractive for a segment of synchrotron experiments.  However a key technical challenge is how to maintain the brightness of these sources while imaging a significant horizontal aperture.  In undulator sources we have the luxury of using only tangentially curved optics for focusing in both horizontal and vertical planes.  For bending magnet sources, to collect a reasonable horizontal aperture, we have to use sagittaly curved optics.  This could be as in a crystal monochromator with a sagittally curved 2nd crystal, or with a sagitally focusing cylindrical or toroidal mirror.  At ALS we decided to adopt toroidal mirrors for our protein crystallography beamlines, due to the robustness of the system.  In order to preserve the brightness, it was found that if we had a system with a tangentially collimating pre-mirror and a toroid mirror downstream of the crystal monochromator, if we focused from infinity in the vertical direction and from the real source in the horizontal direction with 2:1 demagnification,  astigmatic coma aberrations vanished. \inblue{This arrangement was used in ALS superbend beamlines as originally described in  \cite{MacDowell2004}. The appendix in this paper describes how the arrangement leads to the elimination of the primary aberration, astigmatic coma.} At the time, the rms horizontal source size was 100 $\mu$m and the residual aberrations were far less.  Following an upgrade of the ALS in 2013, the horizontal beam size was reduced to 40 $\mu$m  rms \cite{Steier_2014}. The residual aberrations were still tolerable, but cause around a factor of 2 decrease in brightness.  With the current ALS-U project, the new lattice will produce a horizontal beam size of 7 $\mu$m rms.  At this value the residual aberrations of our 2:1 aberration correcting toroidal mirror solution are now longer adequate.

This has led us to consider new types of optics which will preserve the brightness of bending magnets sources, with particular importance for MBA-style 4th generation sources.   
For many applications we need good monochromatization and good tunability.  This is provided by the classical collimated double crystal monochromator.  The task therefore is to design an optical element which can accept light from infinity in the vertical direction and from the real source in the horizontal direction and focus it to a point.  The parabolic mirror produces a beam that is parallel in the vertical direction, so that each infinitesimal segment of the beam will land on a unique part of the mirror.  It is evident therefore that along the tangential direction the mirror should be parabolic, but for rays away from the axis, the limiting shape will be elliptical. 
\inblue{This surface was first described by \cite{McKinneySPIE2009} and described as a Diaboloid, to indicate the probable difficulty in making this type of optical element.  The surface was described as a polynominal, based on a classical optical path function analysis.  This work showed the perfect point to point focusing of the vertically deflecting parabolic cylinder and the Diaboloid, as expected and the benefits for focusing small high brightness beams.  This paper also showed that at the 2:1 demagnification condition, the deviation of the Diaboloid shape from the toroid was at its minimum value.} Although the idea is to replace the toroids in our protein crystallography superbend beamlines with this new type of mirror, the application goes further than this, as now we can demagnify much more strongly than in the previous 2:1 aberration corrected toroid case.  One such example is in a high pressure beamline where currently we focus the beam using the normal 2:1 demagnification system and then demagnify to a few micron focus using a K-B mirror pair.  In an updated system using the diaboloid, the K-B mirror will be redundant, and we will be able to directly focus to around a 3 $\mu$m spot size, increasing flux, increasing image quality and decreasing complexity.

It is important to note that fabrication of such complex surfaces has been impossible, at the slope error level that we require.  For a typical application at ALS, the mirror to source distance is 20 m, and so with a 10 $\mu$m vertical source size, including angle doubling on reflection, an angular error budget of half of the angular size of the source, the tangential slope error tolerance is 0.13 $\mu$rads.  The sagittal slope error tolerance is 18 $\mu$rads.  The tangential tolerance is extremely challenging, but mainly limited by metrology.  Local area correction, and stitching interferometer-based metrology have led to 1D curved optics with significantly smaller slope errors \cite{Yamauchi2002}.  The problem is that current high accuracy metrology techniques cannot work with steeply curved optics.  One possible solution is to use metrology based on computer generated hologram (CGH) reference beams, as now widely used in free form optics fabrication. In addition, we have investigated an approximation to the diaboloid which should be much easier to manufacture and test, but which gives similarly high performance. 

We created a tool in the Oasys environment \cite{codeOASYS} that implements numerically the diaboloid surfaces and its approximations. A new widget is created in Oasys (Section~\ref{sec:oasys}) send the created surface to the ray-tracing code SHADOW \cite{codeSHADOW}. We used this application to make simulations of the ALS beamline 12.2.2 (Section~\ref{sec:beamline}) where the toroid is upgraded with diaboloid and the performances are compared for the present ALS storage ring and for the future upgrade ALS-U. We treated in Section~\ref{sec:scan} a possible upgrade of this beamline using high demagnification and analyze the feasibility of using the simpler parabolic-cone to replace diaboloids. Final remarks are given in Section~\ref{sec:summary}.

\inblue{
\section{Definition and implementation of the diaboloid surface}
\label{sec:DiaboloidEqs}
}

For practical purposes we want to define the diaboloid surface as a function of the focal distances $p,q$ ($p$ is the source-mirror distance, $q$ is the mirror-image distance) and the grazing incidence angle is $\theta$. The diaboloid can be placed in two configurations: i) ``collimating'' to convert a spherical wave into a cylindrical wave (geometrically a point-to-segment focusing, and ii) ``focusing'' to convert a cylindrical wave into a spherical one (or segment-to-point focusing). The latter is the usual configuration in synchrotron beamlines, where the main purpose is to refocus a beam that is collimated in the vertical direction and divergent in the horizontal. 

\begin{figure}\label{fig:frame}
\centering
\includegraphics[width=0.9\textwidth]{figures/diaboloid_frame.png}
\caption{Schematic of the reference frames in use: "mirror-related coordinate system" $(X,Y,Z)$ and "mirror-canonical system" $(x,y,z)$}
\end{figure}

The equation of the diaboloid in the ``collimaing'' configuration takes the form  Ref.~\cite{valSPIE} 
\inred{IT MAY BE DEDUCED IN AN APPENDIX LIKE IN LSBL1440}

\begin{multline}
\label{eqn:diaboloidV}
z(x,y) = q \sin2\theta - 
[ (q \sin{2\theta})^2 + 2p^2 + 2 p q + \\
2 (p + q \cos{2\theta}) y - 
2 (p+q)  \sqrt{x^2 + (y + p)^2}) 
]^{1/2}.
\end{multline}

As expected, at the mirror center position, $x, y$ = 0, the mirror height z = 0. The tangential profile ($x$=0) is a parabola $y=z_0 + z^2/(4f)$ with $z_0=q \sin2\theta$ and focal length $ f = q (1-\cos2\theta) /2$. For small angles ($\cos2\theta\approx 1 - 2\theta^2$) and $|y|\ll q$ gives $z\approx 2 \theta q + \theta y$ with slope in the center $dz/dy=\theta$. The sagittal section ($y$=0) is an ellipse $x^2/a^2 + y^2/b^2=1$ with semiaxes $a=q \sin2\theta$ and $b=a \sqrt{p /(p+q)}$ \inred{REF HOWARD LSBL 1465}. 

For ray tracing calculations it is convinient to obtain the numerical mesh as a function of a mirror-related coordinate system ($X,Y,Z$), with origin in the center of the mirror, one axis ($Y$) tangent to the diaboloid surface origin in the direction of the beam propagation, another axis ($X$) tangent to the diaboloid surface origin in the direction perpendicular of the beam propagation, and the third axis ($Z$) is outward pointing perpendicular to $X$ and $Y$. Therefore the surface in its explicit form needed for being evaluated numerically is $Z(X,Y)$ with $X \in [-W/2, W/2]$ and $Y \in [-L/2, L/2]$ with $L$ the mirror length and $W$ the mirror width. The mirror length is sampled in $N_Y$ points and the mirror width in $N_X$ points, therefore the height or elevation of a point is $Z_{i,j}=Z(X_i,Y_j)$.
In most practical cases the grazing incidence $\theta$ is small, therefore Eq.~\ref{eqn:diaboloidV} can be expressed in the $X,Y,Z$ frame by forcing zero slope at the origin (i.e., detrending the plane $z_{plane}=y \theta$). This is done numerically after evaluating the surface using Eq.~\ref{eqn:diaboloidV}. 
The exact explicit expression of the diaboloid $Z(X,Y)$ can be obtained from the rotation of Eq.~\ref{eqn:diaboloidV} an angle $\theta$ \cite{part2}, resulting in a fourth-degree poynomial equation $F(X,Y,Z)=0$. The explicit equation is therefore obtained by solving this equation for the any point of the $(X_i,Y_j)$ mesh. This is done numerically using the {\tt fqs} python library by N. Krvavika ({\tt https://github.com/NKrvavica/fqs}). 



%The exact expression of the diaboloid in the {\it mirror-related coordinate system} is \inred{discussed in the next Section. It is interesting to describe before an approximated form because i) this is what happened from a historical perspective, and ii) the equations o implement are much simpler.} 

%possible (see  Refs.~\cite{Valeriy2020a}, \cite{Valeriy2020b}) but is extremely long. It has been implemented in a Mathematica code \cite{lacey} to produce the numeric mesh. 

%The diaboloid surface equation takes a simple form \cite{Valeriy2020a} when expressed in a coordinate system ($x,y,z$) that we call {\it canonical-mirror system}. This system has the same origin as ($X,Y,Z$) (the mirror center), the $x$ axis is parallel to $X$, but $y$ is parallel to the direction of the beam after reflection (therefore forming an angle $\theta$ with $Y$), and $z$ is perpendicular to $x$ and $y$ thus also forming an angle $\theta$ with $Z$ (see Fig.~\ref{fig:frame}).



 


%Once we have the equation of the diaboloid in the canonical-mirror reference system, one has to rotate it an angle $\theta$ around the $x$ axis to retrieve the equation in the mirror-related coordinate system. 
%The expresion obtained is extremely long (see Appendix in Ref.~\cite{Valeriy2020b}). It can be defined in Mathematica as an analytical function, ready for numerical calculations of the desired Diaboloid surface profiles.


%It can be shown that the meridional section of the Diaboloid is a parabola, and the sagittal section is an ellipse (Refs.~\cite{Goldberg2020},\cite{Valeriy2020a}).

% \section{Surfaces that approximate the diaboloid}


% \begin{multline}
% \label{eqn:quartic}
% A Z^4 + 
% B(X,Y) Z^3 + 
% C(X,Y) Z^2 + \\
% D(X,Y) Z + E(X,Y) = 0 
% \end{multline}
% where the coefficients contain the dependency in $X$ and $Y$ (see Eqs. 14-18 in \cite{part2}). To obtain the height Z for a given $(X,Y)$ coordinate in the basal plane one has to solve the equation and select the most appropriated solution, as discussed in Section 4.3 of \cite{part2}. This procedure has to be repeatead for each coordinate of the basal plane, which for a 100 x 100 mesh means solving 10$^4$ quartic solutions. 

%The expresion obtained is extremely long (see Appendix in Ref.~\cite{Valeriy2020b}). It can be defined in Mathematica as an analytical function, ready for numerical calculations of the desired Diaboloid surface profiles.


% \subsection{Toroid}\label{sec:toroid}

% 
% \inred{IT MAY BE POSSIBLE TO DEVELOP THE TOROID IN A QUARTIC SURFACE AND SOLVE IT FOR Z(X,Y), AS DONE INTERNALLY IN SHADOW. HOWEVER, NO DIFFERENCES ARE OBTAINED FOR THE CASES ANALYZED USING THE NATIVE TOROID IN SHADOW AND THE NUMERICAL MESH IN EQ.\ref{eqn:toroid}}

\subsection{Approximations of the diaboloid}\label{sec:approximations}

The simplest approximaton of the diaboloid is the toroid, approximated by a doubly bent surface with radii $R_m$ and $R_s$ (in the meridional and sagittal directions, respectively) given by the focusing conditions (Coddington equations):

\begin{equation}
\label{eqn:radii}
\frac{1}{p} = \frac{2 }{\sin\theta R_r};~~~~~~
\frac{1}{p} + \frac{1}{q} = \frac{2\sin\theta}{ R_s}.
\end{equation}

% The tangential circle (at $X=0$), and sagittal circle ($Y=0$) that are tangent to the mirror center ($X$=$Y$=0) will be
% 
% \begin{equation}
% \label{eqn:toroidTS}
% Z(0,Y) = R_t - \sqrt{R_t^2 - Y^2};~~~~~~~~
% Z(X,0) = R_s - \sqrt{R_s^2 - X^2},
% \end{equation}
% and the toroidal surface \inred{in the vicinity of the mirror center is in good approximation} the addition of both profiles: 
% \begin{equation}
% \label{eqn:toroid}
% Z(X,Y) = Z(X,0) + Z(0,Y) = 
% R_t + R_s - \sqrt{R_s^2 - X^2}
% - \sqrt{R_t^2 - Y^2}.
% \end{equation}

An approximation to the diaboloid better than the toroid consist in taking a parabola in the meridinal direction and then adding a circle in the sagittal direction. However, the sagittal circle must have different radius for the different $Y$ coordinates. This surface that will be called here "parabolic-quasicone" has the form \cite{valSPIE}

% 
\begin{multline}
\label{eqn:parabolicCone}
Z(X,Y) = Y \tan\theta - 2 \sec\theta \tan\theta
\sqrt{Y p \cos\theta + p^2} + \\
2 p \sec\theta \tan\theta +
k_1 + k_2 Y - \sqrt{(k_1 + k_2 Y)^2 - X^2},
\end{multline}
where $k_1$ and $k_2$ are
\begin{equation}
k_1 = \frac{p q \cos\theta \sin2\theta}{p+q};~~~~~~
k_2 = \frac{\sin2\theta(q-2p\cos^2\theta)}{2(p+q)}.
\end{equation}
% 
% The variation of the sagittal radius versus \inred{$y$ (Fig.~\ref{fig:frame}) NOTE THE CHANGE!}  (Eq. (2)\ in \cite{Valeriy2020c} )
% \begin{equation}
% \label{eq:sagRadiusApprox}
% R_s(\inred{y}) = \frac{2  p \sin\theta}{p + q} (q - \inred{y})   \sqrt{\inred{y} / p + \cos^2\theta}
% \end{equation}
% is not linear. The variation of the sagittal radius in a cone is linear, therefore, strictly speaking, our approximation of the diaboloid is not a cone\inred{, but a ``quasicone''}. 




The variation of the sagittal radius versus $Y$  (Eq. 35 in \cite{part2}) is approximately linear
\begin{multline}
\label{eq:sagRadius}
R_s(Y) = \frac{2  p \sin\theta \cos\theta}{p + q} (q - Y \cos\theta)   \sqrt{\inred{Y} \cos\theta / p + \cos^2\theta} \\
\approx 
\frac{2p q \cos^2\theta \sin\theta  }{p + q} - 
\frac{\cos\theta \sin\theta (2 p \cos^2\theta - q)}{p + q} Y.
\end{multline}


A first comment is the central sagittal radius $R_s(Y=0)=p q \sin\theta \inred{\cos^2\theta}/ (p+q)$ is very close, but not exact to the one given by the Coddington equation $R_s^c=2 p q \sin\theta / (p+q)$. The second comment is that the variation of the sagittal radius given by Eq.~\ref{eq:sagRadius} is significantly different (see Fig.~\ref{fig:sagittalRadius}) to a "naive" variation generated the applying the Coddington equation at every point in $Y$ (that "sees" the source and image at distances $p'=\sqrt{(-p \cos\theta - Y)^2 + (p \sin\theta)^2}$ and $q'=\sqrt{(q \cos\theta - Y)^2 + (q \sin\theta)^2}$, respectively, and angle $\theta'=\arcsin(p \sin\theta / p')$).


\subsection{Implementation in Oasys and testing}
\label{sec:oasys}

%  different surfaces where the shown equations are used: \inred{Exact diaboloid (solving Eq.~\ref{eqn:quartic}), approximated diaboloid (projected Eq.~\ref{eqn:diaboloidV}), parabolic-quasicone (Eq.~\ref{eqn:parabolicCone}) and toroid (Eq.~\ref{eqn:toroid})}. These surfaces correspond to the Type I (point-to-segment) focusing scheme. The Type II surfaces are calculated by exchanging the $p$ and $q$ values, and flipping the $Y$ array. In the case of diaboloid surfaces, the exact solution is done solving the quartic Eq.~\ref{eqn:quartic}  numerically using the {\tt fqs} python library. The approximated solution performs the passage from the reference frame in Eq.~\ref{eqn:diaboloidV} to ($X,Y,Z$) by detrending a plane as discussed before. Two options are allowed, in the first the plane has an equation $Z=\theta Y$ and in the second one the plane is the result of the best fit of the diaboloid to a plane. We recall that this is an approximation to the diaboloid in the ($X,Y,Z$) frame, but as it will be discussed below, it is usually a good approximation for small grazing angles. A system analyzed in Ref.~\cite{Yashchuk2019} shows a case where this approximation cannot be used. \inred{For practical purposes, one may  always prefer to use the exact solution.}

We have written an Oasys widget to create a numerical mesh describing the diaboloid surface and its approximations. This interface asks for the type of surface to calculate, with different geometries (diaboloid exact, diaboloid approximated, parabolic-quasicone, toroid) and focusing arrangement (point-to-segment or segment-to-point). The interface also allows to remove the toroid from the calculated surface to enhance the aspherical terms. The surface is written in a {\tt hdf5} formatted file, as it is standard for Oasys surfaces. This format allows to input easily the numerical surface into several Oasys applications, like the ray-tracing tool ShadowOui \cite{codeSHADOWOUI}, but also in wave-optics codes (SRW, Wofry). A view of the interface is in Fig.~\ref{fig:widget}.

\begin{figure}\label{fig:widget}
\centering
\includegraphics[width=0.95\textwidth]{figures/widget.png}
\caption{View of the interface to create the numerical sampling of the diaboloid and related surfaces ("Diaboloid" widget in Oasys/Syned) }
\end{figure}

% To use the created surface with the ray-tracing tool, the Diaboloid widget has to be connected to a "Plane mirror" via the "Oasys Surface Data Converter" found in ShadowOui that will automatically convert the file in {\tt hdf5} format to SHADOW format. The reason of using a Plane mirror is because we add the numeric surface to an existing one. If we select the "Plane" the numeric mesh is added until a flat surface $Z(X,Y)$=0. Another possibility is to use the ShadowOui "Toroidal Mirror" and add the Diaboloid but created with the option "detrend Toroid" activated.  

% \section{Testing the diaboloid surfaces by ray-tracing}
% \label{sec:testing}

We performed ray tracing using a single diaboloid with the purpose to test the accuracy of the calculations. The simpler case of point-to-segment configuration is chosen, with $p$=29.3~m, $q$=19.53~m and $\theta$=4.5~mrad. A point source is created with divergence large enough to fully illuminate the mirror dimensions: 
length $L$=0.2~m, and width $W$=0.02~m.

\begin{figure}\label{fig:pointToSegment}
\flushleft
~~~~~~a)~~~~~~~~~~~~~~~~~~~~~~~~~~~~~~~~~~~~~~~~~~~~b)\\
\centering
% \includegraphics[width=0.22\textwidth]{figures/p2s_V.png}
% \includegraphics[width=0.22\textwidth]{figures/p2s_K.png}
% \includegraphics[width=0.22\textwidth]{figures/p2s_parabolic-cone.png}
% \includegraphics[width=0.22\textwidth]{figures/p2s_toroid.png}

\includegraphics[width=0.49\textwidth]{figures/p2s_V_z.png}
\includegraphics[width=0.49\textwidth]{figures/p2s_K_z.png} \\
\flushleft
~~~~~~c)~~~~~~~~~~~~~~~~~~~~~~~~~~~~~~~~~~~~~~~~~d)\\
\includegraphics[width=0.49\textwidth]{figures/p2s_parabolic-cone_z.png}
\includegraphics[width=0.49\textwidth]{figures/p2s_toroid_z.png}
\caption{Comparison of images produced by different surfaces focusing in horizontal and collimating in vertical: a) diaboloid exact (Sect.~\ref{sec:diaboloidexact}), b) diaboloid approximated (Eq.~\ref{sec:diaboloidapproximated}), c) parabolic-quasicone (Eq.~\ref{eqn:parabolicCone}) and d) toroid (Eq.~\ref{eqn:toroid}). \inred{Note the different scales in horizontal and vertical. The Full-Width at Half-Maximum (FWHM) is negligeable in (a), 88 nm in (b), 22 nm for the parabolic-quasicone (c), and 2.5 $\mu$m for the toroid (d).}
}
\end{figure}

The expected result at the focal position is a line focus (segment) with zero dimension in horizontal and a length of $L\sin\theta$ in vertical. The ray-tracing results are shown in Fig.~\ref{fig:pointToSegment} where the focal images for different surfaces are compared: diaboloid (exact), diaboloid (detrended Eq.~\ref{eqn:diaboloidV}) , parabolic-quasicone (Eq.~\ref{eqn:parabolicCone}) and toroid (Eq.~\ref{eqn:toroid}). The intensity profile along the vertical direction has a non-uniform profile due to the fact that the beam has not a uniform distribution when projected on the mirror surface with a grazing angle. As expected, the toroid produces a very aberrated and wide spot (2.5 $\mu$m FWHM and high tails). The parabolic-quasicone reduces the FWHM by two orders of magnitude (22 nm). Aberration tails are observed in this case. The diaboloid produces an almost perfect focusing without aberration tails. However, some residual width is found: 0.15 nm for the exact diaboloid equation and 105 nm for the approximated one. Both values are much below the typical dimensions in practical cases where spots of $\mu$m size or larger are expected because the source size and optical aberrations. For practical purposes both calculations of the diaboloid are good enough.The residual 0.15 nm in the exact implementation is certainly due to numerical round errors and also the use of iterative algorithms to find ray intersects, as done in SHADOW. 

%These numerical errors do not affect the results for practical cases where spots of $\mu$m size or larger are always expected because of the effects of source size and optical aberrations.





% when using the exact implementation (Fig.~\ref{fig:pointToSegment}a) but some 
% residuals are found (88 $\mu$m FWHM) using the approximated implementation (Fig.~\ref{fig:pointToSegment}b) due to errors introduced when performing the basal-plane projection instead of the exact rotation. This residual profile affects only to the width but does not create aberration tails as for Fig.~\ref{fig:pointToSegment}b,c.  

% Indeed, the residual width for the diaboloid also affects to the exact calculation and is studied in detail in Fig.~\ref{fig:mathematica} where we compared the two simulations (exact and approximated) with the diaboloid. The interval of the exact one (Fig.~\ref{fig:mathematica}a) has been zoomed to estimate the width in 0.15 nm FWHM. This residual width is related to several factors: i) truncation errors in the numerical simulations, ii) errors due to the limited number of sampling points, \inred{and iii) errors inherent to the ray tracing implementation. They are discussed in the next paragraph.}


% 
% \begin{figure}\label{fig:mathematica}
% \flushleft
% ~~~~~~a)~~~~~~~~~~~~~~~~~~~~~~~~~~~~~~~~~~~~~~~~~~~~~~~~~~~~~~~~~~~~b)\\
% \centering
% \includegraphics[width=0.48\textwidth]{figures/p2s_exact1.png}
% \includegraphics[width=0.48\textwidth]{figures/p2s_approx1.png}
% \caption{\inred{Detailed focal image produced by a diaboloid calculated using a) the exact method (Section~\ref{sec:diaboloidexact}), and b) the approximated method (Section~\ref{sec:diaboloidapproximated}). The graphics show two different horizontal scale with FWHM values of 0.15 nm (a) and 105 nm (b).}}
% \end{figure}


%Then we propagated one meter downstream the focal line and a rectangle is produced, as we go out of focus in horizontal but in vertical the beam is collimated thus keeping the same length than in the focal position.

%  
% The expected numeric rounding errors are small. The surface evaluation and ray tracing simulations are performed in double precision. The surface is passed from the "Diaboloid widget" to the SHADOW engine via files (one {\tt hdf5} and its translation into SHADOW surface) but the numbers are written in "scientific format". The precision (rounding error) or machine epsilon\footnote{ https://en.wikipedia.org/wiki/Machine\_epsilon} in double precision floating arithmetic is of the order of 2.2 $\times$ 10$^{-16}$. If we work with length units in meters, that corresponds to to less than one femto-second, therefore for our practical purposes, where we work maximum to nano-focusing, this is not a limiting issue. Typically we will use number of points for the $X$ and $Y$ axes in the order of 100-10000 (for our simulations we used 1001 in $Y$ and 101 in $X$). The possible error can be checked by modifying it. With 1001 $\times$ 101 points we get a residual width of 0.15 nm. It is 0.17 nm for 1001 $\times$ 201 and 0.28 nm for 1001 $\times$ 401. Increasing the number of points does not necessarily increase accuracy, but the number of points must be sufficient. If we reduce the number of points to 1001 $\times$ 51 we obtain 0.67 nm, showing the effect of undersampling. Th interpolation used by the ray-tracing implements a polynomial spline. A surface with "enough" points should be used, being limited by the maximum acceptance of SHADOW (500 points in $X$, unlimited in $Y$). 
% 
% Next comment is related to the ray-tracing procedure to compute the intersection of the rays with a surface that is numerically defined. It uses an iterative approximation to get to the "correct" interaction point, and there is an intrinsic error there. From our experience with SHADOW, this has never been a limiting factor, and we do not expect it here as the surfaces are smooth. To check this fact we have compared  diaboloid ray-tracing in two ways, first the usual way to add the numeric mesh on top of a plane mirror, and second to add the diaboloid with a toroid removed on top of a SHADOW Toroid. The heights in the numeric mesh for the second case are smaller thus we expect smaller errors. \inred{We obtained, however, the opposite result: 0.87 nm for the diaboloid detrended with a toroid (vs 0.15 nm for the undetrended case).} As a practical conclusion, it is better to use the diaboloid mesh on top of a plane mirror in SHADOW.
% 
% \inred{After the analysis of all possible sources of error, we can conclude that small errors are due to numerical round errors, sampling points and iterative algorithms to find ray intersects. They contributed to about 0.15 nm in the focal spot (Fig.~\ref{fig:mathematica}a).}
% These numerical errors do not affect the results for practical cases where spots of $\mu$m size or larger are always expected because of the effects of source size and optical aberrations. 
% 
% and iv) in the diaboloid using the approximated method, the errors due to the use of the "basal plane approximation" to rotate the diaboloid from the canonical-mirror system to the mirror-related coordinates (as discussed in Section~\ref{sec:DiaboloidEqs}).}
% 
% There is an significant increase (about two or three orders of magnitude) of the residual errors (105 nm) when the "basal plane-approximation" (Fig.~\ref{fig:mathematica}b). However, these errors are still in the nm regime therefore this simpler method of calculating the diaboloid is still valid for typical optics simulations, where other detected effects are more important.


\section{Ray-tracing the ALS beamline 12.2.2}
\label{sec:beamline}

In this section we analyze the possible use of a diaboloid in a real beamline. We study the case of beamline 12.2.2  at ALS \cite{bl1222} \cite{MacDowell2004}. The source of this beamline is a bending magnet, and we simulated the emission at photon energy of $E$=30~keV. We consider three cases: i) using a source point, ii) the ALS ring
$\sigma_x$ = 26 $\mu$m, $\sigma_y$ = 10 $\mu$m, electron energy $E_e$=1.9 GeV, magnetic field $B$=5.28~T, and iii) the ALS-U ring, with $\sigma_x$ = 10 $\mu$m, $\sigma_y$ = 7 $\mu$m, $E_e$=2.0~GeV, and $B$=3.1~T. We simulated the combined focusing effect of the two beamline mirrors: M1, a plane parabola placed at $p_1$=6.5~m from the source ($L$ = 900~mm, $\theta$=2 ~mrad); and M2, a toroid/diabolod/parabolic-quasicone in a Type II (segment-to-point) configuration at $p_2$=18.8~m from the source ($L$=800~mm, $W$=20~mm, $\theta$ = 2~mrad. The exit slit (focal plane) is at $D$=26.875~m from the source.  The M2 magnification M= $(D-p_2)/p_2$=0.43 is close but not exactly matching the "golden" 1:2 toroid geometry \cite{padmore2000, howells2000}. This beamline uses M1 to collimate the beam in the vertical direction in order to optimize the monochromator performance (not simulated) and then M2 refocus the beam at the entrance slit, therefore M2 sees the source at infinity in vertical and at $p_2$ in horizontal. The possible use of a diaboloid or a parabolic-cone mirror in M2 is studied and the performances compared with the current solution (toroid).

\begin{figure}\label{fig:bl}
\centering
\flushleft
~~~~~a)~~~~~~~~~~~~~~~~~~~~~~~~~~~~~~~~~~~~~~~~~~~~~b) \\
\includegraphics[width=0.45\textwidth]{figures/bl_point_toroid.png} 
\includegraphics[width=0.45\textwidth]{figures/bl_point_parabolic-cone.png} \\
\flushleft
~~~~~c)~~~~~~~~~~~~~~~~~~~~~~~~~~~~~~~~~~~~~~~~~~~~~d) \\
\includegraphics[width=0.45\textwidth]{figures/bl_point_diaboloid.png}
\includegraphics[width=0.45\textwidth]{figures/bl_point_diaboloid-exact.png}
\caption{Focal image produced by a system of two mirrors M1 (collimating parabola) and M2 represented by (a) a toroid, (b) a parabolic-cone, (c) a diaboloid. The FWHM of the horizontal and vertical histograms are given in the plot titles. The scale changes in the different plots. Note the large size given by the toroid (3 $\times$ 45 $\mu$m$^2$) contrary to all the other surfaces (submicron level)
}
\end{figure}

Figure~\ref{fig:bl} shows a source point imaged by the beamline using for M2 a toroid, a parabolic-cone and a diaboloid. The use of a point source helps to address the residual widths for the diaboloid, because it should give zero width. As discussed in the previous section, numeric and sampling errors (due to round errors and iterative calculations of the intercept) produce here a residual of 2.7 $\times$ 7.3 nm$^2$. \inred{It is perhaps also influenced by a negligible but non-zero aberration originated by M1 (a plane parabola) which does not create an exact cylindrical wavefront.}  One can appreciate the high aberrations of the toroid. Two effects can be observed. The intensity profiles produced by the parabolic-cone mirror have aberrated tails not present in the diaboloid.

\begin{figure}\label{fig:als}
\flushleft
a)~~~~~~~~~~~~~~~~~~~~~~~~~~~~~~~~~~~~~~~~~~~b)~~~~~~~~~~~~~~~~~~~~~~~~~~~~~~~~~~~~~~~~~~~c)\\
\centering
\includegraphics[width=0.32\textwidth]{figures/als_toroid.png}
\includegraphics[width=0.32\textwidth]{figures/als_parabolic-cone.png}
\includegraphics[width=0.32\textwidth]{figures/als_diaboloid.png}

\includegraphics[width=0.32\textwidth]{figures/alsu_toroid.png}
\includegraphics[width=0.32\textwidth]{figures/alsu_parabolic-cone.png}
\includegraphics[width=0.32\textwidth]{figures/alsu_diaboloid.png}
\caption{Focal image produced by a system of two mirrors M1 (collimating parabola) and M2 represented by a a) toroid, b) parabolic-cone, and c) diaboloid. The width of the intensity distributions (in FWHM) are written in the graphic titles. 
}
\end{figure}

In Fig.~\ref{fig:als} we compared the images produced by the beamline for a source using the ALS and ALS-U bending magnets. For the ALS case, the diaboloid obviously improves the toroid case currently impemented, but the gain is not dramatic: a factor 2 in vertical FWHM and less in horizontal, which justifies the use of toroids because manufacturing diaboloids is still a challenge. For the ALS-U the gain of using Diabloids is increased to almost a factor 3 in FWHM vertical, which for the toroid is dominated by aberrations. The parabolic-cone is a good approximation as it completely removed the toroid aberrations in vertical, although presenting similar intensity profile in horizontal.

As a conclusion of this section, it is noticeable that for the ALS-U there is a significant improvement in the focal image if the toroid is replaced by a diaboloid. The parabolic-cone is a good approximation of the diaboloid, producing an image of comparable with the diaboloid in terms of FHWM (just a bit larger) but presenting higher tails in the horizontal intensity. 

\section{Comparison of diaboloid and parabolic-quasicone for high demagnification}
\label{sec:scan}

In this section we study a possible upgrade of the beamline to obtain a smaller spot size. For that the magnification of the beamline has to be reduced. We study the beamline presented in the last section for ALS-U where the exit arm of the M2 mirror is reduced to have a smaller magnification M=$q/p$. The position of M2 is not changed therefore the length of the beamline is not constant. We have performed ray-tracing simulations where the M factor is changed to values 1:5 and 1:10 (Fig.~\ref{fig:demagnification}). It can be observed the extremely high aberrations produced by the toroid, which makes its use impossible for these configurations. The diaboloid produces intensity profiles with nice Gaussian profiles, as expected. Interestingly, the parabolic-cone completely removed all the aberration in the toroid presented in the vertical direction producing very good results of beam size FWHM. However, the intensity in horizontal is closer to the toroid than the diaboloid, with a shape that looks more Lorentzian than Gaussian (higher tails). In vertical the parabolic-cone produces a very clean spot, but not as good as the diaboloid. 


\begin{figure}\label{fig:demagnification}
\flushleft
a)~~~~~~~~~~~~~~~~~~~~~~~~~~~~~~~~~~~~~~~~~~~b)~~~~~~~~~~~~~~~~~~~~~~~~~~~~~~~~~~~~~~~~~~~c)\\
\centering
\includegraphics[width=0.32\textwidth]{figures/M0p2_toroid.png}
\includegraphics[width=0.32\textwidth]{figures/M0p2_parabolic-cone.png}
\includegraphics[width=0.32\textwidth]{figures/M0p2_diaboloid.png}
\includegraphics[width=0.32\textwidth]{figures/M0p1_toroid.png}
\includegraphics[width=0.32\textwidth]{figures/M0p1_parabolic-cone.png}
\includegraphics[width=0.32\textwidth]{figures/M0p1_diaboloid.png}
\caption{
Image produced by the beamline for two high demagnification values: 1:5 (top) and 1:10 (bottom) using different M2 shapes: a) toroid, b) parabolic-cone) and c) diaboloid.}
\end{figure}


To study in more detail the aberrations, ray-tracing calculations are performed scanning the magnification factor and extracting the focal dimensions. For each ray-tracing simulation this focal dimension is calculated in two ways: i) the FWHM of the intensity distribution (as discussed previously), and ii) the standard deviation $\sigma$ of the ray coordinates contributing to the focal image. The purpose is to evaluate the aberrations by comparing the FWHM with $\sigma$. If the intensity distribution is Gaussian, we will obtain a $\sigma$ smaller than the FWHM by a factor 2.35. However, when aberrations are present the long tails make that the $\sigma$ increases rapidly becoming even greater than the FWHM value. If this happens it is an indicator of high aberrations.  

\begin{figure}\label{fig:scan}
\flushleft
\centering
a)
\includegraphics[width=0.95\textwidth]{figures/scan_toroid.png}\\
b)
\includegraphics[width=0.95\textwidth]{figures/scan_diaboloid.png}\\
c)
\includegraphics[width=0.95\textwidth]{figures/scan_parabolic-cone.png}

\caption{
Evolution of the (horizontal and vertical) focal size (measured in FWHM and $\sigma$) versus magnification for a beamline with focusing M2 mirror shape: a) toroid, b) diaboloid and c) Parabolic-cone. 
}
\end{figure}


Fig.~\ref{fig:scan} shows the results of ray-tracing calculations of the focal size versus magnification M. Because the size is proportional to the magnification (in the ideal case of zero aberrations), we have visualized the focal size (in FWHM or $\sigma$ for horizontal and vertical directions) divided by the magnification in order to obtain an constant value in the ideal case of zero aberrations. Looking at the response of the toroidal mirror we can identify large aberrations for most cases (the $\sigma$ values are higher than the FWHM) and there is an optimum value around M=0.5 where the vertical size is minimum. This is the "working condition" of most beamlines using toroids at ALS as the aberrations are minimized \cite{padmore2000,howells2000}. For the diaboloid (Fig~\ref{fig:scan}b) the situation is completely different. For most of cases (M$>$0.2) the lines are almost constant and the $\sigma$'s smaller than the FWHM by a value approaching 2.35, indicating that the diaboloid behaves as a perfect optics. For the parabolic-cone (Fig~\ref{fig:scan}c) the situation is very close to the diaboloid when looking at the FWHM values down to the minimum M. When the values of $\sigma$ are higher than FWHM there is presence of residual aberrations. This is true for the horizontal focus, an effect already observed in the previous section, and it is less important for the vertical direction: only for M$<$0.3 the $\sigma$ becomes higher that the FWHM.  

\inred{
\section{Towards the production of approximated diaboloid mirrors}
}

The diaboloid is a highly aspherical surface in both directions, with sagittal section an ellipse and tangential section a parabola. The fabrication of such a surface within the required accuracy required for practical application in an X-ray beamline is a challenging technological problem. 



A realisable approximation of the diaboloid surface could be made by dynamical mechanical bending of a substrate pre-shaped to the sagittal curvature. The dynamical bending would form the tangential shape onto the tangentially flat substrate. This approach is used in many existing toroidal mirrors, where the toroid is obtained by dynamically curving to a circular (tangential) shape a pre-shaped (sagittal) cylinder. By replacing the pre-shape cylinder by a aspheric section, and then apply adequated bending moments one could perhaps approximate well the diaboloid and benefit its superior performances in a beamline. Two solutions may be envisaged for the pre-shaped cone. The first one is to approximate the elliptical sagittal profile to a circular profile that linearly changes along the $Y$ (tangential) direction following the Eq.~\ref{eq:sagittalRadiusLinearized}. A second one, more precise at a first view, would consist in pre-shaping a cone with elliptical (sagittal) section. However, these ellipses are different for different $Y$. 

%This cone degenerates in a cylinder for 2:1 demagnification. Again, this variation is null for 2:1 demagnification becoming a cylinder with elliptical section, which could be envisaged to be manufactured. 

To assess the feasibility of manufacturing a diaboloid-like surface it is convenient to analyze the height differences between the diaboloid and the toroidal surface which is spherical in both directions. With the Oasys Diaboloid widget we can easily substract the best toroid from the diaboloid. In Fig.~\ref{fig:detrended} we have analysed diaboloid surfaces for different magnification (1:5, 1:2 and 1:1) and grazing angles 2 mrad. For all surfaces $p$=20 m and the best toroid has been removed, in order to study the aspherical components. Some sagittal profiles are also shown. It can be appreciated in the magnification 1:5 ($p$=20 m, $q$=4m) and 1:1 ($p$=$q$=20) a slight cariation of the sagittal profile when going from one edge ($Y$=-100 mm) to another ($Y$=100 mm). However the change of height is abour 100 $\mu$m for the first and 6 $\mu$m for the second. As expected (discussed in \cite{part2} Section 6), for 1:2 magnification the sagittal profiles are almost identical as a result of vanishing the linear term in Eq.~\ref{eq:sagittalRadiusLinearized}. This case corresponds to the well-known 2:1 demagnification "golden rule" implemented in the crystallography beamlines at the Advanced Light
Source. These are good candidate to construct a first approximation to the diabloid. 



\begin{figure}\label{fig:detrended}
\flushleft
a)~~~~~~~~~~~~~~~~~~~~~~~~~~~~~~~~~b)~~~~~~~~~~~~~~~~~~~~~~~~~~~~~~c)\\
\centering
\includegraphics[width=0.32\textwidth]{figures/diaboloid_detrended_1:5_image.png} 
\includegraphics[width=0.32\textwidth]{figures/diaboloid_detrended_1:2_image.png} 
\includegraphics[width=0.32\textwidth]{figures/diaboloid_detrended_1:1_image.png} \\
\includegraphics[width=0.32\textwidth]{figures/diaboloid_detrended_1:5_profile.png}
\includegraphics[width=0.32\textwidth]{figures/diaboloid_detrended_1:2_profile.png}
\includegraphics[width=0.32\textwidth]{figures/diaboloid_detrended_1:1_profile.png}

\caption{
Height difference between the diaboloid and the toroid for different magnifications a) 1:5, b) 1:2, c) 1:1. In all simulations $p$=20 m and grazing angle is 2 mrad. The detrended toroid major radii are R$_{a)}$=4 Km, R$_{b)}$=10 Km, R$_{c)}$=20 Km and minor radii r$_{a)}$=13.3 mm, r$_{b)}$=26.7 mm, r$_{c)}$=40 mm.
}
\end{figure}

\begin{figure}\label{fig:detrended5mrad}
\flushleft
a)~~~~~~~~~~~~~~~~~~~~~~~~~~~~~~~~~b)~~~~~~~~~~~~~~~~~~~~~~~~~~~~~~c)\\
\centering
\includegraphics[width=0.32\textwidth]{figures/diaboloid_detrended_5mrad_1:5_image.png} 
\includegraphics[width=0.32\textwidth]{figures/diaboloid_detrended_5mrad_1:2_image.png} 
\includegraphics[width=0.32\textwidth]{figures/diaboloid_detrended_5mrad_1:1_image.png} \\
\includegraphics[width=0.32\textwidth]{figures/diaboloid_detrended_5mrad_1:5_profile.png}
\includegraphics[width=0.32\textwidth]{figures/diaboloid_detrended_5mrad_1:2_profile.png}
\includegraphics[width=0.32\textwidth]{figures/diaboloid_detrended_5mrad_1:1_profile.png}

\caption{
Same as Fig.~\ref{fig:detrended} but for an incident angle of 5 mrad (instead of 2 mrad). The detrended toroid major radii are R$_{a)}$=1.6 Km, R$_{b)}$=4 Km, R$_{c)}$=8 Km and minor radii r$_{a)}$=33.3 mm, r$_{b)}$=66.6 mm, r$_{c)}$=100 mm.
}
\end{figure}



We can now apply this analysis to the case of the beamline 12.2.2 studied before. For the upgraded ALS storage ring, the beamline produced a spot size of 15 $\times$ 57 $\mu$m$^2$ (H $\times$ V) with a toroidal mirror in M2, and this would become 10 $\times$ 18 $\mu$m$^2$ with a diaboloid. As the situation is close to the 2:1 demagnification and the incidence angle is 2 mrad, this is an interesting case for upgrading the mirror to an approximated diaboloid. Fig.~\ref{fig:detrendedBeamline}a show the diaboloid profile with a toroid detrended. 


\begin{figure}\label{fig:detrendedBeamline}
\flushleft
a)~~~~~~~~~~~~~~~~~~~~~~~~~~~~~~~~~b)~~~~~~~~~~~~~~~~~~~~~~~~~~~~~~c)\\
\centering
\includegraphics[width=0.32\textwidth]{figures/diaboloid_bl1222_detrended_image.png} 
\includegraphics[width=0.32\textwidth]{figures/paraboliccone_bl1222_detrended_image.png} 
\includegraphics[width=0.32\textwidth]{figures/ellipticalcylinder_bl1222_detrended_image.png} 


\includegraphics[width=0.32\textwidth]{figures/diaboloid_bl1222_detrended_profile.png}
\includegraphics[width=0.32\textwidth]{figures/paraboliccone_bl1222_detrended_profile.png}
\includegraphics[width=0.32\textwidth]{figures/ellipticalcylinder_bl1222_detrended_profile.png}

\caption{
a) Diaboloid surface with a toroid detrended for BL 12.2.2 and selected sagittal profiles. b) Parabolic quasicone approximation with the toroid detrended. c) Elliptical cylinder bent to a parabola. The detrended toroid major radius is R=8.075 Km and the minor radius is  r=22.595 mm.
}
\end{figure}






Two approximated solutions are studied by ray tracing. First, using a substrate with a circular sagittal section with radius that linearly changes along the $Y$ (tangential) direction Fig.~\ref{fig:detrendedBeamline}b. The second one, more precise at a first view, would consist in pre-shaping a cylinder with elliptical (sagittal) section that is the exact sagittal profile at $Y$=0 (Fig.~\ref{fig:detrendedBeamline}c).  This cylinder with elliptical section could be envisaged to be manufactured with sufficient accuracy. 

The spot size produced by the exact diaboloid is 10 $\times$ 18 $\mu$m$^2$, and 13 $\times$ 25 $\mu$m$^2$ with the first approximation (linearized parabolic quasicone, or cone bent to parabola). In the case that this cone is degenerated into a cylinder, the size becomes 13 $\times$ 33 $\mu$m$^2$ and an aberration tail appears. This is due that this beamline is close but not exactly at 1:2 magnification (it is exactly $p$:$q$=8.075:18.800). If using the second approximation (cylinder with elliptical section) we obtain a similar spot: 15 $\times$ 33 $\mu$m$^2$ and an aberration tail, for the same reason. We check now the beamline in a perfect 1:2 confuguration, by setting $q$=9.4 m. In this configuration, the spot calculated with the diaboloid and the two approximations used are compared in Fig.~\ref{fig:finalcomparison}. The resulting focal sizes are similar for both approximations. In conclusion, approximated sagittal cylinders work well only in 1:2 magnification, and there is no additional benefit to make an elliptical shape with respect to the circular one. For other magnifications the exact diaboloid sshole be used.   




\begin{figure}\label{fig:finalcomparison}
\flushleft
a)~~~~~~~~~~~~~~~~~~~~~~~~~~~~~~~~~b)~~~~~~~~~~~~~~~~~~~~~~~~~~~~~~c)\\
\centering
\includegraphics[width=0.32\textwidth]{figures/final_exact.png} 
\includegraphics[width=0.32\textwidth]{figures/final_approx1.png} 
\includegraphics[width=0.32\textwidth]{figures/final_approx2corrected.png} 

\caption{ Image produced by bl 12.2.2 in exact 1:2 magnification by 
a) exact diaboloid b) Diaboloid approximated by a cylinder (circular section) bent to a parabola c) Diaboloid approximated by a cylinder (elliptical section) bent to a parabola\footnote{in this case the diatance $p$ has been shifted to correct for a small residual aberration tail. Now $q$=9.1 m instead of $q$=9.4 m}. 
}
\end{figure}


% which would require to pre-shape a cylinder with elliptical cross section instead of the current circular shape. The maximum deviation from the circle is about 25 $\mu$m. For the other cases a linearization of the change would be a good simplification, as the sagittal profiles have similar aspect. The situation changes considerably when going to less grazing angle. Less grazing imply less aberrations and therefore the amplitudes of the height differences between the diaboloid and the toroid are only a few microns (Fig.~\ref{fig:detrended5mrad}). However, the shapes of the profiles change when going from $Y$=-100 mm to $Y$=100 mm, meaning that the linearization does not work \inred{TO BE CHECKED...}. 




% Dear  Friends of the Diaboloid,
% Manolo pointed out numerous discrepancies in my last version of the diaboloid note...........mainly that I had confused type 1 and type 2, ie. point to line and line to point.  We need the latter, type 2.  So I went through all again, and expanded a bit.  The deviation of the diaboloid at the magic 2:1 (as Manolo pointed out, the 2:1 is an approximation.....turns out to be a pretty good one) turns out to be only 1.5 microns at the edge of our aperture.  I get this by just subtracting the surface heights, and then also by an expansion method which directly leads to the x^4 height residual.  So the fact that the shape to be made is a plane ellipse, and it doesn't deviate much from a cylinder makes me hope that this shape can be made.  The off 2:1 case, like the 5:1 demag case where the shape is highly conical is much more problematic.  1.5 microns is doable by the way by differential deposition of Si............its the way that APS used to make plane ellipses, starting with a sphere.  I think the APS and NSLS2 have long computer controlled sputtering machines for making these varying thickness coatings.  You cannot do this in metal as the coatings become very rough.  So in principle this could be done by starting with the cylinder and just putting a variable thickness coating down.............this is done for many other apps.   But, we need good interferometric metrology.  



\section{Summary}
\label{sec:summary}

We summarized the basic equations of the diaboloid and its approximations and implemented them in Oasys to perform several ray tracing simulations. 
The "parabolic-quasicone" (Eq.~\ref{eqn:parabolicCone}), an  approximation to the diaboloid, is also implemented. It may play in the future an important role in applying the diaboloid concept to new beamlines, because it is probably easier to manufacture and polish to enough quality. We studied these mirrors surfaces applied to the ALS beamline 12.2.2 in its present (ALS) and in the Upgraded configuration (ALS-U). It is manifested the possible interest to migrate to diaboloid or parabolic-cone solutions for the ALS-U, if they can be built. The use of these new solutions may also improve the beamline performances because they can be used with high demagnification. We showed that magnifications of 1:5 and up to 1:10 may be used with parabolic-cone or diaboloid mirrors for this beamline. 

% We have derived an exact solution of the diaboloid mirror in the form
% \begin{equation}
% \label{eqn:summary_form}
% z(x,y) = - \sqrt{A - B y - C \sqrt{x^2 + (y - D)^2}},
% \end{equation}
% in the coordinate system where the source beam is inclined downward at an angle of $2 \theta$, and $x$ and $y$ are rooted at the mirror center. The complete expression for $z(x,y;p,q,\theta)$ is given in Eq. \ref{eqn:7new_z}. It remains to test this exact expression against existing series representations (from McKinney), and through ray-trace beamline modeling.



     %-------------------------------------------------------------------------
     % The back matter of the paper - acknowledgements and references
     %-------------------------------------------------------------------------

     % Acknowledgements come after the appendices

\section{Acknowledgements}       
 
 
This work was supported by the Director, Office of Science, Office of Basic Energy Sciences, of the U.S. Department of Energy under Contract No. DE-AC02-05CH11231.



     % References are at the end of the document, between \begin{references}
     % and \end{references} tags. Each reference is in a \reference entry.

% \begin{references}
% \reference{Author, A. \& Author, B. (1984). \emph{Journal} \textbf{Vol}, 
% first page--last page.}
% \end{references}
%\cite{knuth84}

%% Note added by Overleaf: If using bibtex, remove the "references" environment above, and uncomment the following line.
\referencelist{iucr}


%      %-------------------------------------------------------------------------
%      % TABLES AND FIGURES SHOULD BE INSERTED AFTER THE MAIN BODY OF THE TEXT
%      %-------------------------------------------------------------------------
% 
%      % Simple tables should use the tabular environment according to this
%      % model
% 
% \begin{table}
% \caption{Caption to table}
% \begin{tabular}{llcr}      % Alignment for each cell: l=left, c=center, r=right
%  HEADING    & FOR        & EACH       & COLUMN     \\
% \hline
%  entry      & entry      & entry      & entry      \\
%  entry      & entry      & entry      & entry      \\
%  entry      & entry      & entry      & entry      \\
% \end{tabular}
% \end{table}
% 
%      % Postscript figures can be included with multiple figure blocks
% 
% \begin{figure}
% \caption{Caption describing figure.}
% \includegraphics{fig1}
% \end{figure}


\end{document}                    % DO NOT DELETE THIS LINE
%%%%%%%%%%%%%%%%%%%%%%%%%%%%%%%%%%%%%%%%%%%%%%%%%%%%%%%%%%%%%%%%%%%%%%%%%%%%%%


